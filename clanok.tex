% Metódy inžinierskej práce

\documentclass[10pt,twoside,slovak,a4paper]{article}

\usepackage[slovak]{babel}
%\usepackage[T1]{fontenc}
\usepackage[IL2]{fontenc} % lepšia sadzba písmena Ľ než v T1
\usepackage[utf8]{inputenc}
\usepackage{graphicx}
\usepackage{url} % príkaz \url na formátovanie URL
\usepackage{hyperref} % odkazy v texte budú aktívne (pri niektorých triedach dokumentov spôsobuje posun textu)

\usepackage{cite}
%\usepackage{times}

\pagestyle{headings}


\title{Zaplatíš a vyhráš. Nevynná zábava alebo hazard?\thanks{Semestrálny projekt v predmete Metódy inžinierskej práce, ak. rok 2022/23, vedenie: Vladimír Mlynarovič}} % meno a priezvisko vyučujúceho na cvičeniach

\author{Martin Jozek\\[2pt]
	{\small Slovenská technická univerzita v Bratislave}\\
	{\small Fakulta informatiky a informačných technológií}\\
	{\small \texttt{...@stuba.sk}}
	}

\date{\small 31.10.2022} % upravte



\begin{document}

\maketitle

\begin{abstract}
\ldots
Pay-to-win hry, alebo ‘zaplať a vyhráš hry’ sú v dnešnej dobe jedny z najhranejších na svete. Možno medzi ne zaradiť hry, ako napríklad World of Tanks, Clash Royal, Clash of clans a mnoho ďalších.

Pokrok v týchto hrách sa može zdať niektorým ľuďom veľmi pomalý, a preto sa nejaká časť rozhodne urýchliť svoj postup, a to zaplatením za vylepšenia, nové postavy, rôzne itemi. No je tu časom možnosť, že nevinné vrážanie peňazí sa može zvrhnúť až ku gamblingu (hazardnému hraniu). 

A presne o tomto bude aj tento článok. Okrem výhod a nevýhod týchto hier sa zameriame aj na to, či sú tieto hry vyslovene vytvorené k hazardu, alebo ide len o nevynnú možnosť rýchlejšieho rastu v hre.
\end{abstract}


\section{Čo je to Pay-to-win?} \label{2}
Výraz Pay-to-win (ďalej už len zaplať a vyhráš)  sa stal veľmi nepopulárnym výrazom v hernom svete. Používatelia a hráči sú schopní vďaka investovaniu pár eur/dolárov odomknúť si rôzne vylepšenia, ktoré sú bežne dosažiteľné až v neskoršej fáze hrania. Platby sú neobmedzené frekvenciou a hodnotou, sú priamo spojené s konkurencieschopnosťou hráčov alebo s pokrokom v hre. No môže to byť taktiež prezentované ako nespravodlivosť vo svete online komunity.

Presnejšiu definíciu poskytuje článok  (1) :"Zaplať a vyhráš hry definujeme ako videohry, ktoré ponúkajú možnosť zaplatiť za obsah, predmety, postavy, ktoré pomáhajú hráčovi napredovať v hre. Zatiaľčo voliteľný nákup vyššie spomenutých vecí je bežnou súčasťou online hier. Produkty zaplať a vyhráš poskytujú výraznú   výhodu v hre a výrazne zvyšujú pravdepodobnosť výhry."

\subsection{O akých typoch hier je reč?} \label{2.1}
Najvýznamnejšími odvetviami sú bojové, strategické, RPG (hry v ktorých každý hráč má nejakú rolu) hry. Ako už bolo skôr uvedené  medzi zaplať a vyhraj hry možno zaradiť World of Tanks, Clash of clans, ale napríklad aj menej populárna hra Genshin Impact a mnoho ďalších.

\section{Výhody/Nevýhody} \label{3}


Základné výhody:

\begin{itemize}
\item rýchly pokrok
\item istota výhry
\item zážitok z hry 
\end{itemize}

Ten istý zoznam, len číslovaný:

\begin{itemize}
\item veľké množstvo investovaných peňazí
\item psychické problémy - závislosť
\item sklon ku gamblingu
\end{itemize}


\section{Gambling a závislosť} 



\section{Herné veľryby}




\section{Ziskovosť} 




\section{Záver}  % prípadne iný variant názvu
\section{Literatúra} 
\end{document}

