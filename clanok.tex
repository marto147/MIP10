% Metódy inžinierskej práce

\documentclass[10pt,twoside,slovak,a4paper]{article}

\usepackage[slovak]{babel}
%\usepackage[T1]{fontenc}
\usepackage[IL2]{fontenc} % lepšia sadzba písmena Ľ než v T1
\usepackage[utf8]{inputenc}
\usepackage{graphicx}
\usepackage{url} % príkaz \url na formátovanie URL
\usepackage{hyperref} % odkazy v texte budú aktívne (pri niektorých triedach dokumentov spôsobuje posun textu)

\usepackage{cite}
%\usepackage{times}

\pagestyle{headings}


\title{Zaplatíš a vyhráš. Nevynná zábava alebo hazard?\thanks{Semestrálny projekt v predmete Metódy inžinierskej práce, ak. rok 2022/23, vedenie: Vladimír Mlynarovič}} % meno a priezvisko vyučujúceho na cvičeniach

\author{Martin Jozek\\[2pt]
	{\small Slovenská technická univerzita v Bratislave}\\
	{\small Fakulta informatiky a informačných technológií}\\
	{\small \texttt{xjozek@stuba.sk}}
	}

\date{\small 6.11.2022} % upravte



\begin{document}

\maketitle

\begin{abstract}
\ldots
Pay-to-win hry, alebo ‘zaplať a vyhráš hry’ sú v dnešnej dobe jedny z najhranejších na svete. Možno medzi ne zaradiť hry, ako napríklad World of Tanks, Clash Royal, Clash of clans a mnoho ďalších.

Pokrok v týchto hrách sa može zdať niektorým ľuďom veľmi pomalý, a preto sa nejaká časť rozhodne urýchliť svoj postup, a to zaplatením za vylepšenia, nové postavy, rôzne itemi. No je tu časom možnosť, že nevinné vrážanie peňazí sa može zvrhnúť až ku gamblingu (hazardnému hraniu). 

A presne o tomto bude aj tento článok. Okrem výhod a nevýhod týchto hier sa zameriame aj na to, či sú tieto hry vyslovene vytvorené k hazardu, alebo ide len o nevynnú možnosť rýchlejšieho rastu v hre.
\end{abstract}


\section{Čo je to Pay-to-win?} \label{2}
Výraz Pay-to-win (ďalej už len zaplať a vyhráš)  sa stal veľmi nepopulárnym výrazom v hernom svete. Používatelia a hráči sú schopní vďaka investovaniu pár eur/dolárov odomknúť si rôzne vylepšenia, ktoré sú bežne dosažiteľné až v neskoršej fáze hrania. Platby sú neobmedzené frekvenciou a hodnotou, sú priamo spojené s konkurencieschopnosťou hráčov alebo s pokrokom v hre. No môže to byť taktiež prezentované ako nespravodlivosť vo svete online komunity.

Presnejšia definícia:
Zaplať a vyhráš hry definujeme ako videohry, ktoré ponúkajú možnosť zaplatiť za obsah, predmety, postavy, ktoré pomáhajú hráčovi napredovať v hre. Zatiaľčo voliteľný nákup vyššie spomenutých vecí je bežnou súčasťou online hier. Produkty zaplať a vyhráš poskytujú výraznú   výhodu v hre a výrazne zvyšujú pravdepodobnosť výhry.\cite{1}

\subsection{O akých typoch hier je reč?} \label{2.1}
Najvýznamnejšími odvetviami sú bojové, strategické, RPG (hry v ktorých každý hráč má nejakú rolu) hry. Ako už bolo skôr uvedené  medzi zaplať a vyhraj hry možno zaradiť World of Tanks, Clash of clans, ale napríklad aj menej populárna hra Genshin Impact a mnoho ďalších.

\section{Výhody/Nevýhody} \label{3}


Základné výhody:

\begin{itemize}
\item rýchly pokrok
\item istota výhry
\item zážitok z hry 
\end{itemize}

Ten istý zoznam, len číslovaný:

\begin{itemize}
\item veľké množstvo investovaných peňazí
\item psychické problémy - závislosť
\item sklon ku gamblingu
\end{itemize}


\section{Gambling a závislosť} 

\subsection{Čo je to gambling?}
Gambling je forma činnosti, v rámci ktorej sa hrá o peniaze a výsledok tejto hry zavisí viac menej od šťastia alebo náhody. Pravdepodobnosť výhry je ovplyvnená formou jednotlivých hazardných hier, pri opakovanej hre je však táto činnosť zisková pre prevádzkovateľa, nie pre hráča. Jednoducho povedané: čím viac hier hráč opakuje, tím je väčšia strata hazardného hráča. To je práve dôvod, prečo existuje svet hazardného priemyslu.

\subsection{Čo je to závislosť?}
Z medicínskeho hľadiska je závislosť liečiteľné chronické ochorenie. Závislosť je nutkavá, chronická, fyziologická alebo psychologická potreba návykovej látky, správania alebo činnosti so škodlivými fyzickými, psychickými alebo sociálnymi účinkami. Po vysadení návykovej latky, činnosti, potreby sa pravidelne vyskytujú známe symptóny ako napríklad úzkosť, podráždenosť, nevoľnosť alebo nespavosť.\cite{4}


\section{Gambling v zaplať a vyhráš hrách}
V zaplať a výhráš hrách je možné "vyhrať aj klasickým pomalým štýlom bez platenia. Postupne sa vypracovať. No tieto hry sú často priamo navrhnuté tak, aby povzbudili hráčov k plateniu za itemi, vylepšenia.

Taktiež hra je navrhnutá tak, aby hráč do hry investoval aj čas a pociťoval počas hrania pre neho pozitívne emócie. Je to veľmi jednoduché, zo začiatku je jednoduché prechádzať levelmi hry, no po nejakom čase hra začína byť zložitá. A keďže hráč si vytvoril tzv. vťah s danou hrou, je veľmi veľká šanca že bude ochotný si priplatiť za možnosť postupu. 

Napríklad špeciálne eventy. Málokedy dokáže hráč prejsť v hre nejaký špeciálny event na prvý pokus. Zo začiatku je to veľmináročné, ale ak zaplatíte v hre nejakú hotovosť, aby ste mohli znova bojovať, s každým ďalším pokusom to bude jednoduchšie.

Už len na základe doteraz spomenutých vecí sa dá povedať že zaplať a vyhraj hry môžu ľudí navádzať k závislosti na hrání a neskôr aj ku gamblingu.


\section{Herné veľryby}
Herná veľryba je pomenovanie pre hráča, ktorý je schopný utratiť, alebo ktorý utratil veľké množstvo peňazí, aby urýchlili svoj postup v hre, alebo aby sa dostali na pomyselný "vrchol rebríčka".

A práve pre kreátorov hier, kde sa vyskytuje možnosť zaplatenia za výhru, sú veľryby presne to, čo chcú.  Herné veľryby nemýňajú veľké sumy naraz. Ide skôr o počet mikrotransakcií, ktorých je veľmi veľa. Sčítaním súm z mikrotransakcií sa dá dostať až na astronomické číslo. A časom sa môžu mikrotransakcie dostať až na platby v hodnote niekoľkých stovák eur/dolárov.

V podstate celý vývojový koncept zaplať a vyhráš hier viedol presne k tomuto. Vytiahnuť z hráčov peniaze ak sa chcú zlepšovať.\cite{3}



\section{Ziskovosť} 
V súčastvosti časť herných spoločností používa model zaplať a vyhráš aby maximalizovali svoj zisk. 
Asi najlepším priíkladom je spoločnosť EA (Electronic Arts), ktorá sa znaží vytlačiť čo najväčší zisko zo svojej hráčskej základne. Podľa GameSpot EA dosialhla tržby vyše 5miliárd dolárov, pričom zisko bol viac ako 1miliarda.

EA stanovuje štandard, že spoločnosti môžu zahrnúť model zaplať a vyhráš do svojich hier a zároveň nečeliť negatívnym finančným následkom. Hráči ďalej môžu len očakávať, že aj ďalšie spoločnosti sa inšpirujú a zakomponujú tento model aj do svojich hier.\cite{2}


\section{Záver} 
Ako bolo spomenuté v článku, Prílišným investovaním, hraním sa može teoreticky vytvoriť závislosť na hraní, chuti stále vyhrávať, a taktiež ide aj istú formu gamblingu. Herné spoločnosti aj napriek kritike z radov hráčov tento model z hier neodstraňujú, práve naopak, posilňujú. 

Zaplať a vyhráš v rozumnej miere, tzn. kým sa človek nedostane do štádia závislosti, nemusí byť zlý model hry. No práve kvôli tomu, že hráči počas hrania, môžu stratiť pojem o realite, môže byť tento model nebezpečný pre hráča.
\section{Literatúra} 
\bibliography{literatura.bib}
\bibliographystyle{plain} 

\end{document}

